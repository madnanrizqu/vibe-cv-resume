%-------------------------
% Resume Preamble - Gergely Template
% Based on Pragmatic Engineer style
% License : MIT
%------------------------

\usepackage{latexsym}
\usepackage[empty]{fullpage}
\usepackage{titlesec}
\usepackage{marvosym}
\usepackage[usenames,dvipsnames]{color}
\usepackage{verbatim}
\usepackage{enumitem}
\usepackage[colorlinks=true,linkcolor=blue,urlcolor=blue]{hyperref}
\usepackage{fancyhdr}
\usepackage[english]{babel}
\usepackage{tabularx}

% fontawesome
\usepackage{fontawesome5}

% fixed width
\usepackage[scale=0.90,lf]{FiraMono}

% Colors
\definecolor{light-grey}{gray}{0.83}
\definecolor{dark-grey}{gray}{0.3}
\definecolor{text-grey}{gray}{.08}
\definecolor{heading-blue}{RGB}{26, 115, 232}

\DeclareRobustCommand{\ebseries}{\fontseries{eb}\selectfont}
\DeclareTextFontCommand{\texteb}{\ebseries}

% custom underline
\usepackage{contour}
\usepackage[normalem]{ulem}
\renewcommand{\ULdepth}{1.8pt}
\contourlength{0.8pt}
\newcommand{\myuline}[1]{%
  \uline{\phantom{#1}}%
  \llap{\contour{white}{#1}}%
}

% custom font: helvetica-style
\usepackage{tgheros}
\renewcommand*\familydefault{\sfdefault}
\usepackage[T1]{fontenc}

\pagestyle{fancy}
\fancyhf{}
\fancyfoot{}
\renewcommand{\headrulewidth}{0pt}
\renewcommand{\footrulewidth}{0pt}

% Adjust margins
\addtolength{\oddsidemargin}{-0.625in}
\addtolength{\evensidemargin}{-0.625in}
\addtolength{\textwidth}{1.25in}
\addtolength{\topmargin}{-.5in}
\addtolength{\textheight}{1.0in}

\urlstyle{same}

\raggedbottom
% REMOVED \raggedright for justified text
\setlength{\tabcolsep}{0in}

% Sections formatting - Blue color for section headings
\titleformat{\section}{
  \color{heading-blue}\bfseries \vspace{-8pt} \raggedright \large
}{}{0em}{}[\color{black} \titlerule \vspace{-5pt}]

%-------------------------
% Custom commands

\newcommand{\resumeItem}[1]{
  \item\small{
    {#1 \vspace{-1pt}}
  }
}

% THREE-COLUMN layout for gergely style:
% Row 1: Job Title (left) | Company (center, absolute) | Date (right)
% Row 2: Location (centered)
\newcommand{\resumeSubheading}[4]{
  \vspace{1pt}
  \item
    \noindent
    \rlap{\textbf{#1}}%
    \hfill
    \raisebox{0pt}[0pt][0pt]{\makebox[0pt]{\textbf{#2}}}%
    \hfill
    \llap{{\color{dark-grey}\small #3}}\\[2pt]
    \begin{tabular*}{\textwidth}[t]{c}
      {\small #4}\\
    \end{tabular*}
}

\newcommand{\resumeSubSubheading}[2]{
    \item
    \begin{tabular*}{\textwidth}{l@{\extracolsep{\fill}}r}
      \textit{\small#1} & \textit{\small #2} \\
    \end{tabular*}\vspace{-7pt}
}

\newcommand{\resumeProjectHeading}[1]{
    \item
    {#1}
    \vspace{-7pt}
}

\newcommand{\resumeSubItem}[1]{\resumeItem{#1}\vspace{-4pt}}

% Single-line entries for education/simple entries
\newcommand{\resumeSimpleEntry}[3]{
  \item
    \begin{tabular*}{\textwidth}[t]{l@{\extracolsep{\fill}}r}
      \textbf{#1}, #2 & {\color{dark-grey}\small #3}\\
    \end{tabular*}\vspace{-7pt}
}

% Filled round bullet for items
\renewcommand\labelitemii{$\vcenter{\hbox{\tiny$\bullet$}}$}

\newcommand{\resumeSubHeadingListStart}{\begin{itemize}[leftmargin=0in, label={}]}
\newcommand{\resumeSubHeadingListEnd}{\end{itemize}\vspace{0.25pt}}
% Filled bullet style (●)
\newcommand{\resumeItemListStart}{\begin{itemize}[leftmargin=0.15in, label=$\bullet$]}
\newcommand{\resumeItemListEnd}{\end{itemize}}

\color{text-grey}
